\cabstract{
    \par 随着计算机技术的快速发展,以及互联网公司的不断增多,人们获取信息的方式和速度发生了巨大的变化。在日常生活的过程中,计算机使用者将会面对大量的文件,其中以数字媒体文件(例如图像、音频、视频等)居多。

    \par 本文设计了一种文件分类管理软件,并进行了开源实现。后文称此软件为“本软件”。本软件包括软件主体和插件。其中,软件主体包括一个基于事件驱动的插件管理器,这使得本软件具有高可拓展性。插件部分包括 16 个插件,具有文件信息查看和管理、相似相同文件检索、标签预测和管理、基于集合运算表达式的文件检索、文件归类整理等功能。基于本软件是自由软件的性质,结合软件工程的理论和实践经验,选择了迭代模型(Iterative model)作为本软件的开发过程模型。

    \par 本文从软件工程的逻辑顺序出发,描述了本软件的开发流程。其中,对于详细设计与实现部分,解释了实现过程中涉及到的数学、密码学、编译、算法竞赛等领域的原理。

}{感知散列\quad 标签\quad 插件管理\quad 自由软件}