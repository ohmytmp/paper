\chapter{总结与展望}

本文描述了一种高可拓展性文件分类管理软件(即本软件)的设计与开源实现。

经过几个月的努力,本文基本撰写完成,其本软件功能也满足了本文中的需求。在完成本文和本软件的过程中,为了实现软件功能,查询了相关资料,参考了相关开源项目。

\section{对开源社区作出的贡献}

在阅读开源项目的源代码、部署运行开源项目时,分别于 Python 模块粒子群匹配、基于粒子群优化的图片处理开源项目——高损图像压缩算法、Python 模块 playsound、Rao.Pics 系列开源项目 rua 中发现了错误(Bug)。

考虑到笔者当时的知识储备,对前端技术十分生疏,因此笔者仅向开源项目 rua 报告了问题(Issue)。同时,笔者对 Python 和粒子群优化略有了解,于是对前三者进行了修改,并提交了代码合并请求(Pull request, PR)。其中,前二者的代码合并请求已经合并(Merge)到项目主分支(Branch)中。Python 模块 playsound 的代码合并请求至今仍未回复,该项目及其唯一管理员已近两年无任何活动记录,笔者考虑将修改后的该模块自行打包上传到 Pypi(Python Package Index,Python 软件包索引,是 Python 官方的软件库)(此行为均符合 Python 模块 playsound 的开源许可证)。

\section{比较与分析}

本软件采用 Python 为内核编程语言,Python 和 C++ 为插件编程语言,设计并实现了文件分类管理程序,事件驱动的插件系统使其具有高可拓展性。

本软件实现了文件信息分析和查看、标签生成件归类整理、相同相似文件检索、集合运算语句查询和文件展示、子图匹配等功能。通过使用本软件,用户可以节省大量的时间。

经过完善的测试流程,本软件满足需求分析所提出的软件需求,形成了一款完整的软件产品。但同时,本软件还存在着许多不足。本软件为个人开发,软件的开发和维护进度将会相对缓慢,同时用户量不足,难以形成有效的社区,缺乏用户反馈。本软件缺少具备完整功能的 GUI,当前 GUI 仅有查询功能。本软件的标签预测插件准确度较低,预测的结果往往需要手动检查、修改。由于 Python 解释器的限制,本软件内核在文件数量达到一定的数量(十万及以上)级时,仍存在性能问题,这个问题同时体现在运行速度和内存占用上。

为进一步讨论本软件的可取之处与不足,以规划未来的开发和维护目标,比较借鉴其它相似软件对本软件的发展具有重要的意义。如表 \ref{table:cmp} 所示,本软件与上文提到的多个相似软件进行了软件性质、平台支持、功能支持和实现方式等方面的比较 \cite{eagle} \cite{pixcall} \cite{billfish} \cite{digikam}。

\begin{table}[h!]
\centering
\begin{tabular}{|c|ccccc|} 
    \hline
    软件 &
    本软件 & digiKam & 
    Eagle & Pixcall & Billfish\\ 
    \hline
    软件性质 &
    自由软件 & 自由软件 &
    专有软件 & 专有软件 & 专有软件 \\
    \hline
    开发者 &
    个人 & 开源社区 &
    商业团队 & 商业团队 & 商业团队 \\
    \hline
    数据管理 &
    可选 & 仅索引 &
    复制 & 仅索引 & 可选 \\
    \hline
    GUI &
    只读 & 完整功能 &
    完整功能 & 完整功能 & 完整功能 \\
    \hline
    Windows &
    原生支持 & 原生支持 &
    原生支持 & 原生支持 & 原生支持 \\
    \hline
    MacOS &
    原生支持 & 原生支持 &
    原生支持 & 原生支持 & 原生支持 \\
    \hline
    Linux &
    原生支持 & 原生支持 &
    第三方支持 & 不支持 & 不支持 \\
    \hline
    移动设备 &
    不支持 & 不支持 &
    不支持 & 不支持 & 支持 \\
    \hline
    网页 &
    支持 & 不支持 &
    第三方支持 & 支持 & 不支持 \\
    \hline
    完整 API &
    支持 & 不支持 &
    第三方支持 & 不支持 & 不支持 \\
    \hline
    数据上云 &
    不支持 & 不支持 &
    不支持 & 支持 & 支持 \\
    \hline
\end{tabular}
\caption{本软件与相似软件的比较}
\label{table:cmp}
\end{table}

\section{软件规划}

本软件作为个人需求驱动的项目,将会随着需求的变化而迭代更新,不会随着本文工作周期的完成而终止。目前,本软件已于开源社区发布源代码,包括十几个仓库(Repository)和数百次提交(Commit),其中内核和部分可选插件已于 PyPi 发布 \cite{ohmytmp}。

除了表 \ref{table:cmp} 中出现的软件外,Celum、Canto、Bynder 等素材管理软件也都支持数据上云功能,个人数据可以在其公司管理的服务器上存储 \cite{celum} \cite{canto} \cite{bynder}。但基于本软件作为自由软件的性质,考虑到隐私保护问题和法律问题,本软件未来也不会提供数据上云或类似功能。

本软件的 GUI 将随着软件的迭代更新逐步开发。目前,独立开发者 MeetQY Rao 基于 Eagle 开发的 Rao.Pics 系列开源项目正在逐步添加 Pixcall 和 Billfish 的支持 \cite{rao}。未来可以考虑对 Rao.Pics 系列开源项目提供代码或直接修改有关项目(上述行为均符合有关项目的开源许可证),增加对本软件的支持。但是,在本文的工作周期内不进行额外的考虑。

本软件的预测功能不够完善,准确度较低,需要结合前沿技术的发展,开发更优秀的预测插件。

\section{展望}

随着编译技术的发展,Python 解释器的性能和 C++ 语法的易用性将会进一步优化。随着深度学习,特别是以卷积神经网络为代表的神经网络方法在模式识别、自然语言处理等领域取得越来越多的成果,并不断拓展新的适用方向,深度学习库将更加易用化、民用化、标准化,本软件基于深度学习的插件将会更加易于开发和使用。随着计算机性能的提升和硬件加速技术的发展,代码运行的效率将得到提升,可以开发更加复杂的软件,或大幅度增强高性能需求的程序的实时性。

可以展望,未来的文件管理软件将更加自动化,预测过程更加高效,预测结果更加准确,插件开发更加方便。
