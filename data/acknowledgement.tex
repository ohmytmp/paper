\acknowledgement{
    \par 在毕业论文里,“致谢”大概是最特殊的存在:不受格式所拘,充满真情实感;见证了我的成长,代表着一路走来的心迹,也蕴藏着一段难忘的人生经历。那些在困厄中给予支持的人,我感谢你的方式,是把你写进我的论文里。当然,因为我的一些隐私习惯,这里同样不会出现真实姓名。我想要感谢的人,看到我的致谢,自然会明白,我说的是谁。

    \par 我想要感谢我的父母,没有你们就没有现在的我。

    \par 我想要感谢本文的指导老师,感谢您在我论文写作期间的指导建议和细心的解答。我想要感谢我的两位导师,感谢您们在我科研路上的指导。

    \par 我想要感谢一门荣誉课程的老师,您带我更加深入地认识和了解了数学领域,解答了我的许多问题,无论是数学上的还是工作、生活上的。您对我来说,更像是一位朋友。我要感谢一门综合性硬件课程、一门中外合作软件课程、一门教材缺失 API 的课程、四门高度涉及 C/C++ 的课程、一门抽象的数学课程、二门极其困难的选修课程的老师。虽然由于课程的性质,可能在学习过程中存在着痛苦,也可能最终没有娶得理想的成绩。但您们的教学水平、教学质量和教学态度,提高了我在该课程上的学习效率,满足了我对一个优秀的课程的想象。

    \par 我要感谢我的室友,开源狂魔 Ciel,技术潮男 iFan,多才多艺 YuhangQ,游戏达人 Renge,勇攀高峰 Magi。我们各不相同,却组成了同一个 4047。我要感谢和我有着相同情况的两位同学,我们相互帮助、相互鼓励,未来三年也请多多关照。我要感谢我的班长,你为我们付出了许多。
    
    \par 我要感谢我的网友爱莉,时常和我进行语 C,平时只向我传递正面情绪,放松了我的心情。我要感谢差一点成为我上司的 Rimo Chan 在粒子群优化相关的领域给予的指导。我要感谢开源社区的朋友们,给我提供了一个良好的技术交流环境,和你们进行的讨论对本文的内容和格式帮助很大。

    \par 我要感谢我的学校。平心而论,我校待我不薄。我要感谢 Overleaf 和 Google Scholar,你们对本文的写作有很大的帮助。我要感谢 Reol、Hanser、Mili、Namewee、Bandai-Namco、KOEI-TECMO,我在写论文的过程中主要听了你们创作的歌曲。

    \par 最后,我要感谢本文的读者,也就是你。感谢你阅读本文。如何你存在任何意见、建议或问题,欢迎向我提出(本文的最后一个引用即为本软件的开源项目的主页,其中的项目 paper 为本论文,在 Issue 提出即可)。如果本文或本软件能为你的生活和工作提供一些微小的帮助,我将不胜荣幸。

    \par 以下为本文致谢名单(排名不分先后,以字符串(UTF-8 编码)字典序为准)(未提供昵称者则用名姓首字母大写代替,如刘看山(Kanshan Liu)为 \tt{K L}):

    \begin{center}
    \begin{tabular}{ccccc}
        10011010 & 10min & Bandai Namco & billchen & C H \\
        Ciel & D C & dad & Elysia\textasciitilde♪ & F G \\
        FSF \& GNU & Google Scholar & H C & H Q & H Z \\
        Hanser & Ice & iFan & Inno Aiolos & J C \\
        jiafeng5513 & JLU & KOEI TECMO & Magi & MeetQY \\
        Merlyn & Mili & mom & Namewee & Overleaf \\
        PSF & Renge & Reol & Rimo Chan & S G \\
        S S & Sakura & Sakura shem & sticker time bot & Superbia \\
        W C & X B & X G & Y C & Y Z \\
        YuhangQ & Zeping Lee & ごとう ひとり & 意义 & 打cs的cs \\
        渣女喵 & 爱莉 & 黑染7 & \\
    \end{tabular}
    \end{center}

    \par And you.
}
